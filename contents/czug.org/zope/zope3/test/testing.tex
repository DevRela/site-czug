\chapter{Testing}
\label{chap:Testing}

\section{Unit testing}

This chapter will discuss about unit testing and integration testing.
Doctest-based testing is heavily used in Zope 3.  And test driven
development (TDD) is prefered in Zope 3.

To explain the idea, consider a use case. A module is required with a
function which returns ''Good morning, name!''.  The name will be
given as an argument.  Before writing the real code write the unit
test for this.  In fact you will be writing the real code and it's
test cases almost in parallel.  So create a file named example1.py
with the function definition:

{\footnotesize
\begin{verbatim}
def goodmorning(name):
    "This returns a good morning message"
\end{verbatim}
}

See, you have not yet written the logic. But this is necessary to run
tests successfully with failures!. Ok, now create a file named
\emph{example1.txt} with test cases, use reStructuredText format:

These are tests for \emph{example1} module.

First import the module:

{\footnotesize
\begin{verbatim}
  >>> import example1
\end{verbatim}
}

Now call the function \emph{goodmorning} without any arguments:

{\footnotesize
\begin{verbatim}
  >>> example1.goodmorning()
  Traceback (most recent call last):
  ...
  TypeError: goodmorning() takes exactly 1 argument (0 given)
\end{verbatim}
}

Now call the function \emph{goodmorning} with one argument:

{\footnotesize
\begin{verbatim}
  >>> example1.goodmorning('Jack')
  'Good morning, Jack!'
\end{verbatim}
}

See the examples are written like executed from prompt.  You can use
your python prompt and copy paste from there.  Now create another file
\emph{test\_example1.py} with this content:

{\footnotesize
\begin{verbatim}
import unittest
import doctest

def test_suite():
    return unittest.TestSuite((
        doctest.DocFileSuite('example1.txt'),
        ))

if __name__ == '__main__':
    unittest.main(defaultTest='test_suite')
\end{verbatim}
}

This is just boilerplate code for running the test. Now run the test
using \emph{python2.4 test\_example1.py} command. You will get output with
following text:

{\footnotesize
\begin{verbatim}
File "example1.txt", line 16, in example1.txt
Failed example:
    example1.goodmorning('Jack')
Expected:
    'Good morning, Jack!'
Got nothing
\end{verbatim}
}

Now one test failed, so implement the function now:

{\footnotesize
\begin{verbatim}
def goodmorning(name):
    "This returns a good morning message"
    return "Good morning, %s!" % name
\end{verbatim}
}

Now run the test again, it will run without failures.

Now start thinking about other functionalities required for the
module. Before start coding write about it in text file. Decide API,
write test, write code, than continue this cycle until you finish your
requirements.  

\subsection{Running tests}

By conventions your test modules are put in tests module under each
package.  But the doctest files can be placed in the package itself.
For example if the package is \emph{ticketcollector}.  Then the main
doctest file can be placed in \emph{ticketcollector/README.txt}.  And
create a sub-package \emph{zopetic.tests}, under this package create
test modules like \emph{test\_main.py}, \emph{test\_extra.py} etc.

To run the unit tests, change to instance home:

{\footnotesize
\begin{verbatim}
$ cd ticketcollector
$ ./bin/test
\end{verbatim}
}
